% Options for packages loaded elsewhere
\PassOptionsToPackage{unicode}{hyperref}
\PassOptionsToPackage{hyphens}{url}
%
\documentclass[
]{article}
\usepackage{amsmath,amssymb}
\usepackage{iftex}
\ifPDFTeX
  \usepackage[T1]{fontenc}
  \usepackage[utf8]{inputenc}
  \usepackage{textcomp} % provide euro and other symbols
\else % if luatex or xetex
  \usepackage{unicode-math} % this also loads fontspec
  \defaultfontfeatures{Scale=MatchLowercase}
  \defaultfontfeatures[\rmfamily]{Ligatures=TeX,Scale=1}
\fi
\usepackage{lmodern}
\ifPDFTeX\else
  % xetex/luatex font selection
\fi
% Use upquote if available, for straight quotes in verbatim environments
\IfFileExists{upquote.sty}{\usepackage{upquote}}{}
\IfFileExists{microtype.sty}{% use microtype if available
  \usepackage[]{microtype}
  \UseMicrotypeSet[protrusion]{basicmath} % disable protrusion for tt fonts
}{}
\makeatletter
\@ifundefined{KOMAClassName}{% if non-KOMA class
  \IfFileExists{parskip.sty}{%
    \usepackage{parskip}
  }{% else
    \setlength{\parindent}{0pt}
    \setlength{\parskip}{6pt plus 2pt minus 1pt}}
}{% if KOMA class
  \KOMAoptions{parskip=half}}
\makeatother
\usepackage{xcolor}
\usepackage[margin=1in]{geometry}
\usepackage{graphicx}
\makeatletter
\def\maxwidth{\ifdim\Gin@nat@width>\linewidth\linewidth\else\Gin@nat@width\fi}
\def\maxheight{\ifdim\Gin@nat@height>\textheight\textheight\else\Gin@nat@height\fi}
\makeatother
% Scale images if necessary, so that they will not overflow the page
% margins by default, and it is still possible to overwrite the defaults
% using explicit options in \includegraphics[width, height, ...]{}
\setkeys{Gin}{width=\maxwidth,height=\maxheight,keepaspectratio}
% Set default figure placement to htbp
\makeatletter
\def\fps@figure{htbp}
\makeatother
\setlength{\emergencystretch}{3em} % prevent overfull lines
\providecommand{\tightlist}{%
  \setlength{\itemsep}{0pt}\setlength{\parskip}{0pt}}
\setcounter{secnumdepth}{-\maxdimen} % remove section numbering
\usepackage{fancyhdr} \usepackage{graphicx}
\ifLuaTeX
  \usepackage{selnolig}  % disable illegal ligatures
\fi
\usepackage{bookmark}
\IfFileExists{xurl.sty}{\usepackage{xurl}}{} % add URL line breaks if available
\urlstyle{same}
\hypersetup{
  pdftitle={Bericht Tresquintos: Chile},
  hidelinks,
  pdfcreator={LaTeX via pandoc}}

\title{Bericht Tresquintos: Chile}
\usepackage{etoolbox}
\makeatletter
\providecommand{\subtitle}[1]{% add subtitle to \maketitle
  \apptocmd{\@title}{\par {\large #1 \par}}{}{}
}
\makeatother
\subtitle{\href{https://tresquintos.cl}{tresquintos.cl/popularidad}}
\author{}
\date{\vspace{-2.5em}aktualisiert: 2024-12-02}

\begin{document}
\maketitle

\addtolength{\headheight}{1.0cm} 
\fancypagestyle{plain}{} 
\pagestyle{fancy} 
\fancyhead[L]{\includegraphics[width = 25pt]{pc.png}}
\fancyhead[R]{\includegraphics[width = 30pt]{logo.png}}
\renewcommand{\headrulewidth}{0pt}

\textbf{Zusammenfassung}: Gabriel Boric ist seit 997 Tagen im Amt. Das
entspricht ungefähr 68.2\% seiner Amtszeit (2022 - 2026). Das
untenstehende Diagramm zeigt einige aktuelle Entwicklungen seiner
Popularität. Die schwarze Linie stellt den monatlichen Durchschnitt dar.
Die grüne Linie zeigt den linearen Trend. Die gestrichelten roten Linien
zeigen den nichtlinearen Trend. Die tägliche Veränderung beträgt 0.01\%.
Aktuell hat Boric eine Zustimmung von 28.9\% und eine Ablehnung von
62.7\%. \textbf{Im Durchschnitt}: Die Zustimmung von Boric sinkt sich um
-0.3\% pro Monat, während seine Ablehnung steigt sich um 0.3\% pro
Monat. Für weitere Details klicken Sie bitte
\href{https://tresquintos.cl/popularidad}{\textbf{hier}}.

\begin{figure}

{\centering \includegraphics[width=0.49\linewidth,height=0.49\textheight]{Plots/chile-pres-aprueba} \includegraphics[width=0.49\linewidth,height=0.49\textheight]{Plots/chile-pres-desaprueba} 

}

\end{figure}

\let\thefootnote\relax

\footnote{**WARNUNG**. Dieser Bericht wird automatisch erstellt, wenn auf dem Server Unterschiede festgestellt werden. Er wird nicht von einem menschlichen Operator überwacht und kann Fehler enthalten. Für Kommentare oder Fragen, senden Sie bitte eine E-Mail an: comunicaciones@tresquintos.cl.}

\end{document}
